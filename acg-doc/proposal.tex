\documentclass[preprint]{acmsiggraph}          % preprint

\usepackage[scaled=.92]{helvet}
\usepackage{times}

%% The 'graphicx' package allows for the inclusion of EPS figures.

\usepackage{graphicx}

%% use this for zero \parindent and non-zero \parskip, intelligently.

\usepackage{parskip}

%% Optional: the 'caption' package provides a nicer-looking replacement
%% for the standard caption environment. With 'labelfont=bf,'textfont=it',
%% caption labels are bold and caption text is italic.

\usepackage[labelfont=bf,textfont=it]{caption}

%% If you are submitting a paper to the annual conference, please replace 
%% the value ``0'' below with the numeric value of your OnlineID. 
%% If you are not submitting this paper to the annual conference, 
%% you may safely leave it at ``0'' -- it will not be included in the output.

\onlineid{0}

%% Paper title.

\title{Parallel Toon Shading on the BlueGene/L}

%% Author and Affiliation (single author).

%%\author{Roy G. Biv\thanks{e-mail: roy.g.biv@aol.com}\\Allied Widgets Research}

%% Author and Affiliation (multiple authors).
\author{%
Ben Boeckel\thanks{e-mail: boeckb@rpi.edu} %
\and Artem Kochnev\thanks{e-mail: kochna@rpi.edu} %
\and Abhishek Mukherjee\thanks{e-mail: mukhea2@rpi.edu} %
\and Taro Omiya\thanks{e-mail: omiyat@rpi.edu}}

%% Keywords that describe your work.

\keywords{parallel, toon, cel-shading, bluegene}

%%%%%% START OF THE PAPER %%%%%%

\begin{document}

\maketitle

\begin{abstract}

In the growing times where software are moving towards parallelism, new schemes
have to be reprogrammed to utilize multiple core powers.  Toon shading, a
non-photo-realistic rendering popularized by Sega's Jet Grind Radio, would
particularly benefit from this movement.  The graphical technique requires a
number of pre-calculation based on the model's properties, such as material and
a specialized one-dimensional texture.  With this in mind, a parallel system
would very easily optimize cell shading rendition to its best efficiency.

While numerous works created their own unique graphical style, we'll be working
with the ``watercolor'' style seen in games: cel-shading (also known as hard
shading) without bold outlines.  There are a few examples of this style in
games, including Grasshopper Manufacture's No More Heroes and Nintendo's Legend
of Zelda: The Wind Waker.  Due to the time constraint, the outline rendering
common in cel-shaded games will not be implemented.  As our mentioned games
shown, humans are still capable of perceiving depth on cel-shaded graphics
without borders.  Shadows will be used as the depth indicator instead, and if
we have enough time, we may add atmospheric perspective as well.

Goal-wise, we'll initially begin with a simple single dimension texture
implementation found in the paper below, Stylized Rendering Techniques For
Real-Time 3D Animation, and the shadow rendition, Shadows for Cel Animation.
If time permits, we'd also like to try the two dimensional texture extension
from Xtoon: An Extended Toon Shader and replacing certain degree of shading
with bill boarded texture a method also mentioned in Stylized Rendering
Techniques For Real-Time 3D Animation.  The intended style, though, may differ
from the pencil sketch version the document mentions.

\end{abstract}

%% ACM Computing Review (CR) categories. 
%% See <http://www.acm.org/class/1998/> for details.
%% The ``\CRcat'' command takes four arguments.

%\begin{CRcatlist}
%  \CRcat{K.6.1}{Management of Computing and Information Systems}%
%{Project and People Management}{Life Cycle};
%  \CRcat{K.7.m}{The Computing Profession}{Miscellaneous}{Ethics}
%\end{CRcatlist}

%% The ``\keywordlist'' command prints out the keywords.
\keywordlist

\section*{Timeline}

\section*{Related Works}

To Robert, for all the bagels.\cite{lake2000srt}

\bibliographystyle{acmsiggraph}
\nocite{*}
\bibliography{parashader}
\end{document}

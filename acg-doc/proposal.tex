\documentclass{acmsiggraph}

\usepackage[scaled=.92]{helvet}
\usepackage{times}

%% The 'graphicx' package allows for the inclusion of EPS figures.

\usepackage{graphicx}

%% use this for zero \parindent and non-zero \parskip, intelligently.

\usepackage{parskip}

%% Optional: the 'caption' package provides a nicer-looking replacement
%% for the standard caption environment. With 'labelfont=bf,'textfont=it',
%% caption labels are bold and caption text is italic.

\usepackage[labelfont=bf,textfont=it]{caption}

%% If you are submitting a paper to the annual conference, please replace 
%% the value ``0'' below with the numeric value of your OnlineID. 
%% If you are not submitting this paper to the annual conference, 
%% you may safely leave it at ``0'' -- it will not be included in the output.

\onlineid{0}
\newcommand{\timestamp}{\textbf}

%% Paper title.

\title{Parallel Ray Tracing on the BlueGene/L}

\author{%
Ben Boeckel\thanks{e-mail: boeckb@rpi.edu} %
\and Artem Kochnev\thanks{e-mail: kochna@rpi.edu} %
\and Abhishek Mukherjee\thanks{e-mail: mukhea2@rpi.edu} %
\and Taro Omiya\thanks{e-mail: omiyat@rpi.edu}}

\keywords{parallel, raytracing, bluegene}

\begin{document}

\maketitle

\begin{abstract}
While libraries such as Nvidia's CUDA can greatly optimize the graphical
application, it's pipeline structure causes inaccuracies to occur in lighting
physics.  As such, an efficient, accurate graphical application is required
Ray-tracing can render lightings very accurately, but falls short in efficiency
With a massively parallel system such as Blue Gene, however, ray-tracing can be
rendered at a much faster speed.
\end{abstract}
\keywordlist

\section{Introduction}
Several attributes aboute ray-tracing lends itself well to networked computers
such as the Blue Gene L, regardless of whether it uses threads or Message
Passing Interface (MPI).  Each ray in ray-tracing acts independently from each
other, and communication between each calculation occurs only in the end of
tracing.  Finally, ray-tracing can be rendered using the CPU, only.

Through this project, we will prove that it is possible to generate realistic
graphics on a highly parallel system created originally for scientific uses.
The libraries we plan to use is the MPI library for the Blue Gene L.
The results will be a png image file.  With time, we may implement other
effects, such as a limited version of non-photo realistic implementation,
different reflection/refraction effects, and others.

The program will be tested using the object models from Advanced Computer
Graphics.  Various measurements will be used to test the performance of our
program: the number of processors, number of models and texture, number of time
rays can bounce, and number of samples for shadows.

\section*{Timeline}
\begin{itemize}
% TODO: Fill int
\item Test
\end{itemize}

\section*{Related Works}

\begin{itemize}
\item
\textbf{Stylized rendering techniques for scalable real-time 3d animation}.
An incredibly popular paper describing three distinct techniques on rendering
flat, cartoon-like graphics efficiently.The first technique implements a
flexible method of shading flat colors, using a one dimensional texture and a
simple calculation used on all vertexes.It crudely rounds the surface and
lighting direction normals to map shades to the model.The second method uses
the same implementation as the first technique, except using separate textures
as shades rather than color.The third part describes a method of creating
bold outlines on sharp edges to further imply flatness on the
model.\cite{lake2000srt}
\end{itemize}

\bibliographystyle{acmsiggraph}
\bibliography{parashader}
\end{document}

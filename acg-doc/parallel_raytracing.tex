\documentclass{acmsiggraph}

\usepackage[scaled=.92]{helvet}
\usepackage{times}

%% The 'graphicx' package allows for the inclusion of EPS figures.

\usepackage{graphicx}

%% use this for zero \parindent and non-zero \parskip, intelligently.

\usepackage{parskip}

%%Taro: inserting urls

\usepackage{url} 

%% Optional: the 'caption' package provides a nicer-looking replacement
%% for the standard caption environment. With 'labelfont=bf,'textfont=it',
%% caption labels are bold and caption text is italic.

\usepackage[labelfont=bf,textfont=it]{caption}

%% If you are submitting a paper to the annual conference, please replace 
%% the value ``0'' below with the numeric value of your OnlineID. 
%% If you are not submitting this paper to the annual conference, 
%% you may safely leave it at ``0'' -- it will not be included in the output.

\onlineid{0}
\newcommand{\timestamp}{\textbf}

%% Paper title.

\title{Parallel Ray Tracing on the BlueGene/L}

\author{%
Ben Boeckel\thanks{e-mail: boeckb@rpi.edu} %
\and Artem Kochnev\thanks{e-mail: kochna@rpi.edu} %
\and Abhishek Mukherjee\thanks{e-mail: mukhea2@rpi.edu} %
\and Taro Omiya\thanks{e-mail: omiyat@rpi.edu}}

\keywords{parallel, raytracing, bluegene}

\begin{document}

\maketitle


\begin{abstract}
While libraries such as Nvidia's CUDA can greatly optimize the graphical
application, it's pipeline structure causes inaccuracies to occur in lighting
physics.  As such, an efficient, accurate graphical application is required
Ray-tracing can render lightings very accurately, but falls short in efficiency
With a massively parallel system such as Blue Gene, however, ray-tracing can be
rendered at a much faster speed.
\end{abstract}
\keywordlist


\section{Introduction}
Several attributes aboute ray-tracing lends itself well to networked computers
such as the Blue Gene L, regardless of whether it uses threads or Message
Passing Interface (MPI).  Each ray in ray-tracing acts independently from each
other, and communication between each calculation occurs only in the end of
tracing.  Finally, ray-tracing can be rendered using the CPU, only.

Through this project, we will prove that it is possible to generate realistic
graphics on a highly parallel system created originally for scientific uses.
The libraries we plan to use is the MPI library for the Blue Gene L.
The results will be a png image file.

The program will be tested using the object models from Advanced Computer
Graphics.  Various measurements will be used to test the performance of our
program: the number of processors, number of models and texture, number of time
rays can bounce, and number of samples for shadows.

In addition, a few extensions were added to create several filters similar to
those found in image editors such as Adobe Photoshop. With a little tweaking, we
can give graphics a bit of an artistic taste.


\section{Related Works}
Several papers has already delved into the topic of ray tracing for highly
parallel systems.  One notable work is from Benthin and his work will ray
tracing on the Cell processor, most commonly found in Sony's Playstation 3.
His careful consideration in the architecture of the processor demonstrates the
importance of the system's structure in calculations. \cite{benthin2006rtc}

A more thorough analysis is found in Badouel's paper, where he mentions the
specific structures, algorithms, and varies strategies to avoid costly data
transfers on the Blue Gene L.  He uses scheduling and data managing to most
effectively balance each processors' workload and minimize latency.  Furthermore,
careful ways of avoiding common parallel problems, such as deadlocks, are
addressed from this paper.  This will act as the main basis for our project. 
\cite{badouel1994dda}


\section{Parallelism}
TODO


\section{Filters}
In an image editor, a filter is an algorithm that converts the pixel values to
a new pixel to either remove unwanted artifacts or add an artistic taste to an
image.  Often, in an image, a pixel is reperesented by a red, green, and blue
value each corresponding to additive colors in lights.  Using these values,
we can recalculate and compile a new image that gives a different impression
from the original.

In this case, we created a gray-scale filter and a color-limiting filter
that recalculates the generated pixel value from each ray in the ray-tracer.


\section*{Gray Scale}
Each pixel in a gray scale image is reperesented, frequently, by a single
gamma value that reperesents a shade of gray.  Finding the gamma from an
RGB value is, incidently, very simple.  We simply have to find the weighted
acerage of each value:

$\gamma = W_{r}R + W_{g}G + W_{b}B$

Where $1 = W_{r} + W_{g} + W_{b}$

One can easily convert this back to the RGB format by setting the red,
green, and blue value equal to \gamma.  It's worth noting that the weight
values must be carefully chosen, as it reperesents each color's contribution
to the image.  For example, if we naively give equal weights to red, green,
and blue, we get an overly-lit image.
%TODO: add 2 images: original_color.png, overly_lit_gray_scale.png

To create the most acceptable color-to-gray-scale conversion, we used
the weight values from
\url{www.mathworks.com/support/solutions/data/1-1ASCU.html}.

$W_{r}R = 0.299$

$W_{g}G = 0.587$

$W_{b}B = 0.114$
%TODO: add normal_gray_scale.png


\section*{Limiting Color}
Many image editors includes an option to posterize an image, rendering
a group of near-colored pixels to be shaded in one color.  While our program
can not detect neighboring pixels, it can generalize colors to limited shades
by calculating where it fall in the spectrum of the RGB value.

Limiting the color spectrum is fairly easy.  We divide the color spectrum evenly to
the number of shades the user wants.  If a pixel's value falls under any of the middle
sections, we set it to a pre-computed value corresponding to that portion.  The only
exception is towards the two ends, where they will be set to either 0 or maximum color
value.
%TODO: add 2 images: red_scale.png, red_shades.png

The basic algorithm can be described as follows:

\textit{Let $D$ be the quotient of maximum color value, $M$, divided by $n$ number of shades.}

\textit{Let $S = M / (n - 1)$.}

\textit{For index $i$ between 0 and $n$}

\hspace{10 mm} \textit{If color red ($r$) is less than $i \times D$, set $r = S \times i$ }

\textit{Repeat for green ($g$) and blue ($b$).}
%TODO: add color_shades.png


\section{Conclusion}
TODO

Since every pixel value had no ability to reference from their consecutive neighbors,
our filters had to be solely based on a single pixel value.  If we can devise a way to
read neighboring pixel values, we could implement a more intelligent, visually pleasing
filters to more varied artistry work.


\section{Citation}

\begin{itemize}
\item
\textbf{Distributing data and control for ray tracing in parallel} \cite{badouel1994dda}
\item
\textbf{Ray tracing on the cell processor} \cite{benthin2006rtc}
\item
\end{itemize}

\bibliographystyle{acmsiggraph}
\bibliography{parashader}
\end{document}

